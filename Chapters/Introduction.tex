

{\fontsize{20}{26}\selectfont\textbf{Introduction}}


Agriculture has long been the backbone of human civilization, yet it faces mounting pressures that threaten its sustainability and efficiency. One such challenge lies in the harvesting of delicate and high-value crops such as asparagus sprouts, a task that demands precision, adaptability, and real-time decision-making. Traditionally, this process relies on human laborers who painstakingly identify, pick, and place each sprout into collection bins, a labor intensive endeavor constrained by fatigue, rising costs, and a shrinking workforce. This situation underscores a critical problem: the absence of intelligent, scalable systems capable of replicating human dexterity and judgment in the unpredictable, unstructured environment of a field. To address this gap, this thesis proposes an autonomous robotic system driven by advanced technologies: a 3D stereo camera to generate point cloud data, a neural network to classify sprout features,  to interpret complex data and control a robot arm for picking and placing asparagus sprouts. This integration of robotics, computer vision, and artificial intelligence aims to revolutionize agricultural automation.


The necessity of this research is rooted in the urgent challenges facing modern agriculture. As the global population rises to 10 billion by 2050, the demand for food production intensifies, while the availability of arable land and labor decline. Asparagus, valued for its nutritional properties and economic return, exemplifies these difficulties: its sprouts require selective harvesting based on size, maturity, and condition, a task ill-suited to rigid, reprogrammed robotic systems. Labor shortages, driven by urbanization, aging rural populations, and economic pressures, further exacerbate the issue, leaving farmers struggling to maintain productivity. This system offers a solution by reducing the dependence on human labor, enhancing efficiency, and ensuring consistent crop quality. By processing 3D point-cloud data from stereo cameras, classifying sprout attributes with neural networks, and leveraging LLMs for adaptive decision making and robotic control, this approach tackles the complexity of field-based harvesting with unprecedented sophistication.


This work is more relevant today than ever because of the convergence of technological advancements and societal needs. Recent breakthroughs in LLMs, such as their ability to process multi modal data and generate context-aware responses, have opened up new possibilities for real-time decision making in dynamic environments. Coupled with improvements in affordable 3D imaging and robotic hardware, these tools enable a level of autonomy previously unattainable. Meanwhile, climate change and resource scarcity increase the urgency for sustainable farming practices, making intelligent automation a critical tool for resilience. In contrast, achieving this feat was prohibitively difficult in the past. Earlier robotic systems lacked the computational power and sensory integration needed to navigate the variability of natural settings: mud, uneven terrain, and irregular crop growth confounded simplistic algorithms. Computer vision struggled with poor resolution and limited depth perception, while decision-making models were too rigid to adapt to unforeseen conditions. The absence of LLMs meant that that robots could not interpret complex, unstructured data or learn from diverse inputs, rendering them ineffective for tasks requiring human-like judgment.

The benefits of this system extend widely, promising trans-formative impacts for humanity and agriculture. For farmers, it reduces labor costs, mitigates workforce shortages, and boosts productivity, enabling them to meet growing food demands efficiently. By automating repetitive, physically demanding tasks, it improves worker well-being and reallocates human effort to strategic roles like crop planning. Environmentally, it supports sustainability by optimizing harvest timing, reducing waste, and minimizing resource use. On a broader scale, this technology could reshape agricultural practices, offering a blueprint for automating other crops and fostering resilient food systems. By bridging robotics, vision, and LLMs, this thesis not only solves a specific challenge but also pioneers a future where intelligent machines enhance our capacity to feed the world.



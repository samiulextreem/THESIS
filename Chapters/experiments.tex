\chapter{Experiments}
This chapter details the empirical evaluation of the perception and manipulation pipeline in both benchtop and integrated robot settings. 
Our objectives are to: (i) verify that the trained detector produces reliable, actionable targets; (ii) validate the grasping and 
cutting strategy on representative specimens; and (iii) quantify end-to-end performance under realistic operating conditions.






\section{Experiments with robot arm}
\subsection*{Objective and setup}
Integrated experiments assess end-to-end performance when the perception module, camera, and 2-DOF robot arm operate together. The camera 
is rigidly mounted and calibrated using the intrinsic and hand--eye procedures described earlier. The work surface replicates bed geometry, 
and safety interlocks limit speed and workspace during testing.

\subsection*{Method}
Upon detection, target poses are transformed into the robot base frame. A Cartesian approach is planned that maintains tool alignment 
with the spear centerline and enforces a minimum clearance from neighboring spears. After grasping, the tool lifts to a safe height. 
For cutting trials, a cut point is computed at a fixed height above the soil plane and executed with a guarded motion.

\subsection*{Evaluation}
Each run records detection latency, planning latency, execution time, total cycle time, and task outcome (grasp-only or grasp-and-cut). 
We also log near-miss events (clearance violations prevented by the controller) and any operator interventions.

\subsection*{Results overview}
The integrated system demonstrated reliable perception-to-action handoff, with stable cycle times and accurate tool placement under the 
tested conditions. The perception module maintained high detection quality (see Computer Vision chapter) and provided targets that resulted 
in smooth approach trajectories. Typical bottlenecks were motion planning under tight clearances and occasional re-plans due to dynamic 
shadows or reflections. Qualitative examples of successful runs are included in the accompanying figures.









\section{Grasping experiments}
\subsection*{Objective and setup}
Grasping trials evaluate the ability of the end-effector to securely grasp and stabilize individual asparagus spears without
damage. Experiments were conducted on a lab bench with fresh spears placed in soil trays to approximate field support conditions.
The camera and lighting followed the configuration described in the Computer Vision chapter to ensure consistency between 
perception and manipulation trials.

\subsection*{Method}
For each scene, the detector infers spear instances and corresponding masks. A grasp pose is generated by fitting a centerline to the instance mask and selecting a grasp band at a fixed offset above the soil plane. Candidate poses are filtered by thickness and verticality constraints, and the highest-confidence feasible pose is dispatched to the controller. The gripper closes to a torque-limited setpoint before lifting slightly to test retention.

\subsection*{Protocol}
\begin{enumerate}
  \item Place a single or multiple spears in varied orientations and spacings.
  \item Acquire an image, run inference, and compute grasp pose(s).
  \item Execute the grasp using the preset approach trajectory and grasp force.
  \item Record success/failure, grasp time, and any visible damage.
  \item Repeat over mixtures of scenes to cover occlusions and illumination changes.
\end{enumerate}

\subsection*{Metrics}
We report grasp success rate (percentage of trials in which the spear is secured and maintained for 3 seconds), visible-damage rate, 
mean time-to-grasp (from image capture to stable grasp), and pose-planning success (percentage of scenes where a feasible grasp pose 
is produced).

\subsection*{Findings}
Across the collected trials, the system consistently generated feasible grasp poses and executed stable grasps on most scenes. 
Failure modes primarily involved severe occlusions near the grasp band or excessive surface moisture causing slippage. Representative 
outcomes and grasp overlays are provided in the results figures.


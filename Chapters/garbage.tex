he arm operates with a servo gear reduction ratio of
30:1, a choice validated by the design’s final verdict as sufficient to handle the
required torques while maintaining energy efficiency, critical for prolonged
field operations. Each joint is powered by NEMA 17 motors, selected for
their compact size and robust performance, delivering a rated torque of .4 Nm
with gear reduction , the value stand for 12 Nm, which exceeds the calculated
maximum torque of 3.36 Nm in the arm’s fully stretched position, even after
applying a 30% safety margin to account for the payload, finger tip, and link
weights in worst case scenario. This is 7.26 times more than calculated max
torque required for this applicaiton. The arm features three rotational joints
(R1, R2, R3), with R1 at the base (root revolution) experiencing a torque of
T1 = 2.8 Nm, and R2 bearing a torque of T2 = .8 Nm due to the cumulative
forces from the arm’s extension.



To calcuate the torque, at first we calculate the force at joints –fully stretched–
yield forces atas follows, F2 = .5 Kg(mass of the stepper motor+ mass of
gearbox ) × 9.8 = 4.5 N and F3 = .3 kg (mass of stepper motor) × 9.8
= 2.98 N), resulting in gravitational force of 4.5 N and 2.98 N at R2 and
R3, respectively. And from this force value we can estimate the torque value
required for the motor to rotate. The arm’s weight distribution is optimized
for field mobility: the total weight at joint R2 is 0.46 kg (comprising the 0.3
kg NEMA 17 motor and 0.16 kg of additional components), while R3 weighs
0.3 kg, and the finger tip at the end effector has a maximum weight of 15 g
to ensure minimal load at the arm’s extremity. Lightweight materials such as
wood or aluminum should be chosen for the links, balancing durability with
the need to reduce energy consumption, as heavier materials would strain the
motors and battery life in uneven asparagus fields.


The training dataset comprises 150 annotated images of asparagus fields cap-
tured under diverse conditions, such as varying lighting, weather, and weed
density. Each image includes labeled bounding boxes and segmentation masks
for asparagus spears, prepared in the COCO format to ensure compatibility
with the framework. Initial evaluation is conducted via field tests rather than
a predefined test dataset.
The model runs on a CUDA.raining parameters include a batch size of 2, a learning rate of 0.00025, and
300 iterations—sufficient for initial fine-tuning. The learning rate remains
constant during training to maintain stability.